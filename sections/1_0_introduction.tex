% !TEX encoding = UTF-8
% !TEX TS-program = pdflatex
% !TEX root = ../tesi.tex

%**************************************************************
\chapter{Introduction}\label{ch:introduction}
%**************************************************************

\intro{In this section we will present the summarized content of the whole thesis.}
%\noindent Esempio di utilizzo di un termine nel glossario \\
%\gls{api}. \\
%
%\noindent Esempio di citazione in linea \\
%\cite{site:agile-manifesto}. \\
%
%\noindent Esempio di citazione nel pie' di pagina \\
%citazione\footcite{womak:lean-thinking} \\


% The introduction allows you to orient the reader to your research project and preview the organisation of your thesis. In the introduction, state what the topic is about, explain why it needs to be further researched and introduce your research question(s) or hypothesis.

% Whilst patterns of organisation in introductions vary, there are some common features that will help you to achieve an informative and engaging introduction. Let’s identify these features:

% Introduce the topic
% Define key terms and concepts
% Give background and context for the topic (this may include a brief literature review)
% Review and evaluate the current state of knowledge in the topic (this may include a brief literature review)
% Identify any gaps, shortcomings and problems in the research to date
% Introduce your research question(s) or hypothesis
% Briefly describe your methodology and/or theoretical approach
% Explain the aim of your research and what contribution it will make to the topic
% Give an overview of the chapter outline of the thesis.
% It’s important to note that, depending on your field of study and the faculty requirements of your thesis, not all of these features will be relevant. Also, these features may occur in varied orders.

% Most people write many drafts of their introduction. It can be useful to write one early in the research process to clarify your thinking. You will need to write a version for your confirmation proposal and other milestones. As your research progresses and your ideas develop, you will need to revise it. When the final draft of chapters is complete, check the introduction once more to make sure that it accurately reflects what you have actually done.

% One way to evaluate the robustness of a machine
% learning model is to search for inputs that produce incorrect outputs. Inputs intentionally designed to fool deep learning models are referred to as adversarial examples (Goodfellow et al., 2017).

%**************************************************************

\section{Topic definition}\label{sec:topic-definition}
%**************************************************************

To understand why AI systems are vulnerable to the same weakness, we
must briefly examine how AI algorithms, or more specifically the machine
learning techniques they employ, “learn.” Just like the reconnaissance officers, the machine learning algorithms powering AI systems “learn” by
extracting patterns from data. 

%**************************************************************

\section{Problem statement}\label{sec:problem-statement}
%**************************************************************
% These carefully curated examples are correctly classified by a human observer but can fool a target model, raising serious concerns regarding the security and integrity of existing ML algorithms. On the other hand, it is showed that robustness and generalization of ML models can be improved by crafting high-quality adversaries and including them in the training data.



%**************************************************************

\section{Research question}\label{sec:research-question}
%**************************************************************

Focusing on text classification task, we are motivated to address the following two fundamental research questions (RQs):

\begin{description}
    \item[RQ1:] \emph{Does state-of-the-art attack methods generate adversarial examples that are legible, grammatical, and similar in meaning to the original texts?}
    \item[RQ2:] \emph{How can we craft high-quality adversaries?}
\end{description}



%**************************************************************


\section{Solution}\label{sec:solution}
%**************************************************************
Researchers proposed special adversarial attacks in the text domain in order to maintain semantic consistency and syntactic correctness.
But those methods fail in generating high-quality adversarial examples since they frequently violate linguistic constraints.

This thesis concentrates on the adversarial attacks for text classification, in particular, the attacks based on sentiment analysis datasets, like IMDB \cite{maas-EtAl:2011:ACL-HLT2011} and Rotten Tomatoes \cite{pang-lee:2005a}.
Two state-of-the-art approaches, TextFooler \cite{journals/corr/abs-1907-11932} and BERT-based attack \cite{conf/emnlp/GargR20}, are compared to analyze weaknesses and strengths.

Then, their shortcomings are addressed by proposing a novel method, called SynBA, to generate adversarial examples for text data.
It is a word-level attack that generates adversarial examples by substituting words with candidates that have both a synonymy and contextual relationship with the original token.

The key contributions of this survey can be summarized as follows:
\begin{itemize}
    \item we introduce a simple but strong attack method, SynBA, to quickly generate high-profile utility-preserving adversarial examples that force the target models to make wrong predictions under the white-box setting;
    \item we propose a comprehensive automatic and human evaluation of adversarial attacks to evaluate the effectiveness, efficiency, and utility preserving properties of our system;
    \item we compare the adversarial examples generated by our method with TextFooler and BERT-based attack in terms of semantic similarity, semantic consistency, perturbation rate, success rate, perplexity and execution time.
\end{itemize}


% Our contributions. This survey concentrates on the adversarial
% attack and defense technology in the NLP field and provides a thorough and systematic review. 
%The key contributions of this survey
% can be summarized as follows:
%  We comprehensively and systematically summarize the textual
% adversarial attack and defense technology, elaborating on textual adversarial examples, adversarial attacks on texts, defenses
% against textual adversarial attacks, applications in various NLP
% tasks, and potential development directions in this domain.
%  We categorize current textual adversarial attacks according to
% the semantic granularity at the top level and further classify
% each class into several subclasses depending on the example
% generation strategy. To the best of our knowledge, we are the
% first to regard the example generation strategy as a classification criterion and propose this two-level classification for
% adversarial attacks.

%

% Briefly describe your methodology and/or theoretical approach
% Explain the aim of your research and what contribution it will make to the topic

% from TextFooler

% Our main contributions are summarized as follows:
% • We propose a simple but strong baseline, TEXTFOOLER, to quickly generate high-profile utility-preserving adversarial examples that force the target models to make wrong predictions under the black-box setting.
% • We evaluate TEXTFOOLER on three state-of-the-art deep learning models over five popular text classification tasks and two textual entailment tasks, and it achieved the state-of-the-art attack success rate and perturbation rate. Algorithm 1 Adversarial Attack by TEXTFOOLER
% • We propose a comprehensive four-way automatic and three-way human evaluation of language adversarial attacks to evaluate the effectiveness, efficiency, and utilitypreserving properties of our system.
% • We open-source the code, pre-trained target models, and test samples for the convenience of future benchmarking



%**************************************************************

\section{Thesis organization}\label{sec:thesis-organization}
\begin{description}
    \item[{\hyperref[ch:introduction]{First chapter}}] introduces the general content about thesis and gives a short presentation of the topic, the problem and the solution we propose;

    \item[{\hyperref[ch:background]{Second chapter}}] a deepening about the theoretical foundations used during the stage and the project;

    \item[{\hyperref[ch:methodology]{Third chapter}}] presents the datasets used during for the training and the testing of the model;

    \item[{\hyperref[ch:experimental-results]{Fourth chapter}}] presents the experiments did during to develop the system;

    \item[{\hyperref[ch:final-discussions]{Fifth chapter}}] discusses about the results and possible future developments.
\end{description}
During the drafting of the essay, following typography conventions are considered:
\begin{itemize}
    \item the acronyms, abbreviations, ambiguous terms or terms not in common use are defined in the glossary, in the end of the present document;
    \item the first occurrences of the terms in the glossary are highlighted like this: \gls{word};
    \item the terms from the foreign language or jargon are highlighted like this: \emph{italics}.
\end{itemize}

%**************************************************************

