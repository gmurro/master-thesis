% !TEX encoding = UTF-8
% !TEX TS-program = pdflatex
% !TEX root = ../tesi.tex

%**************************************************************
% Sommario
%**************************************************************
\cleardoublepage
\phantomsection
\pdfbookmark{Abstract}{Abstract}
\begin{flushright}{
	\slshape
	“The algorithms
	that cause AI systems
	to work so well are
	imperfect, and their
	systematic limitations
	create opportunities for
	adversaries to attack. At
	least for the foreseeable
	future, this is just a fact of
	mathematical life.”
	\\ – Marcus Comiter} \\

%  "One day the AIs are going to look back on us the same way we look at fossil skeletons on the plains of Africa. An upright ape living in dust with crude language and tools, all set for extinction."

% - Ex machina
	\medskip

\end{flushright}


\begingroup
\let\clearpage\relax
\let\cleardoublepage\relax
\let\cleardoublepage\relax

\chapter*{Abstract}

With the advent of high-performance computing devices, deep neural networks have gained a lot of popularity in solving many Natural Language Processing tasks. 
However, they are also vulnerable to adversarial attacks, which are able to modify the input text in order to mislead the target model. 
Adversarial attacks are a serious threat to the security of deep neural networks, and they can be used to craft adversarial examples that steer the model towards a wrong decision.

In this dissertation, we propose SynBA, a novel contextualized synonym-based adversarial attack for text classification. 
SynBA is based on the idea of replacing words in the input text with their synonyms, which are selected according to the context of the sentence. 

We show that SynBA successfully generates adversarial examples that are able to fool the target model with a high success rate. 
We demonstrate three advantages of this proposed approach: (1) effective---it outperforms state-of-the-art attacks by semantic similarity and perturbation rate, (2) utility-preserving---it preserves semantic content, grammaticality, and correct types classified by humans, and (3) efficient---it performs attacks faster than other methods.


\endgroup

\vfill

