%**************************************************************
% Acronimi
%**************************************************************
\renewcommand{\acronymname}{Acronimi e abbreviazioni}
%**************************************************************

\newacronym[description={\glslink{aig}{Artificial Intelligence}}]
{ai}{AI}{Artificial Intelligence}
\newglossaryentry{aig}
{
    name=\glslink{ai}{AI},
    text= Artificial Intelligence,
    sort= ai,
    description={Artificial Intelligence is a branch of computer science that aims to machines able to perform tasks that normally require human intelligence, such as visual perception, speech recognition, decision-making, and translation between languages.}
}


\newacronym[description={\glslink{nlpg}{Natural Language Processing}}]
{nlp}{NLP}{Natural Language Processing}
\newglossaryentry{nlpg}
{
    name=\glslink{nlp}{NLP},
    text= Natural Language Processing,
    sort= nlp,
    description={Natural Language Processing is an important field of Artificial Intelligence, linguistics and computer science. It 's about the interactions between computers and human language, in particular, how to program computers to process and analyze large amounts of natural language data.}
}

\newacronym[description={\glslink{cvg}{Computer Vision}}]
{cv}{CV}{Computer Vision}
\newglossaryentry{cvg}
{
    name=\glslink{cv}{CV},
    text=Computer Vision,
    sort= cv,
    description={Computer Vision is the field of study that deals with how computers can gain high-level understanding from digital images or videos. From the perspective of engineering, it seeks to automate tasks that the human visual system can do.}
}

%**************************************************************
% Glossario
%**************************************************************
%\renewcommand{\glossaryname}{Glossario}




\newglossaryentry{word}
{
    name=\glslink{word}{Word},
    text=word,
    sort=word,
    description={Example of a term in the glossary}
}

\newglossaryentry{auto-regressive}
{
    name=\glslink{auto-regressive}{Auto-regressive},
    text=auto-regressive,
    sort=auto-regressive,
    description={A statistical  model is autoregressive if it predicts future values based on past values.}
}

\newglossaryentry{seq2seq}
{
    name=\glslink{seq2seq}{Seq2seq},
    text=seq2seq,
    sort=seq2seq,
    description={Seq2seq is a model that maps a sequence of symbols to another sequence of symbols.}
}

\newglossaryentry{embedding}
{
    name=\glslink{embedding}{Embedding},
    text=embedding,
    sort=embedding,
    description={Embedding is a relatively low-dimensional space into which you can translate high-dimensional vectors. Embeddings make it easier to do machine learning on large inputs like sparse vectors representing words. Ideally, an embedding captures some semantics of the input by placing semantically similar inputs close together in the embedding space.}
}

\newglossaryentry{lexicon}
{
    name=\glslink{lexicon}{Lexicon},
    text=lexicon,
    sort=lexicon,
    description={A lexicon is a component of
    a NLP system that contains information (semantic,
    grammatical) about individual words or word strings.}
}

\newglossaryentry{nn}
{
    name=\glslink{neural network}{Neural network},
    text=neural network,
    sort=neural network,
    description={A neural network  is a computational learning system that uses a network of functions to understand and translate a data input of one form into a desired output, usually in another form. }
}


